\UseRawInputEncoding
\documentclass[a4paper,12pt]{article}


\usepackage[utf8]{inputenc}
\usepackage[T1]{fontenc}
\usepackage[spanish]{babel}
\usepackage{amsmath, amssymb}
\usepackage{graphicx}
\usepackage{listings}
\usepackage{xcolor}


\lstset{
  language=Python,
  basicstyle=\ttfamily\small,
  backgroundcolor=\color{gray!10},
  frame=single,
  breaklines=true,
  showstringspaces=false,
  captionpos=b
}


\title{\textbf{Informe Lab 2: POO en Python}}
\author{Juan Ignacio Carrera Saavedra}
\date{Octubre 2025 -- UCT}

\begin{document}

\maketitle
\tableofcontents
\newpage

\section{Introducción}

La Programación Orientada a Objetos (POO) organiza el código en "objetos" que combinan atributos y metodos. En el siguiente informe se cubriran algunos de los conceptos fundamentales que este paradigma de programación ofrece.

\section{Herencia}

La herencia permite crear clases hijas como se les suele decir, que heredan atributos y métodos de clases padre, especializando su comportamiento.

\begin{lstlisting}[caption={Herencia básica en Python}]
class Vehiculo:
    def transportar(self):
        return "Transportando genericamente"

class Auto(Vehiculo):
    def transportar(self):
        return "Rodando por la carretera"

mi_auto = Auto()
print(mi_auto.transportar())  # Salida: Rodando por la carretera
\end{lstlisting}

\textbf{Ventajas:}
\begin{itemize}
  \item Evita duplicación de código
  \item Crea jerarquías lógicas
  \item Facilita el mantenimiento
\end{itemize}

\section{Polimorfismo}

El polimorfismo permite que diferentes clases respondan distinto ante los mismos metodos.

\begin{lstlisting}[caption={Polimorfismo en acción}]
class Pato:
    def sonido(self): return "Cuac!"

class Vaca:
    def sonido(self): return "Muu!"

animales = [Pato(), Vaca()]
for animal in animales:
    print(animal.sonido())
\end{lstlisting}

\textbf{Beneficio}: Nos ofrece una gran flexibilidad a la hora de tratar a los objetos.

\section{Clases Abstractas}

Las clases abstractas definen métodos que las subclases deben implementar, se usa como plantilla para otras subclases. Una clase abstracta no se puede instanciar directamente.

\begin{lstlisting}[caption={Clase abstracta con implementación}]
from abc import ABC, abstractmethod

class Forma(ABC):
    @abstractmethod
    def area(self): pass

class Circulo(Forma):
    def __init__(self, radio):
        self.radio = radio
    def area(self):
        return 3.1416 * self.radio ** 2

c = Circulo(5)
print(f"Area: {c.area()}")  # Area: 78.54
\end{lstlisting}

\textbf{Fórmula del círculo:}
\[
A = \pi r^2
\]

\section{Interfaces}

Una interfaz es un contrato que define qué métodos debe implementar una clase, pero no cómo debe implementarlos. En Python se pueden usar con Protocol, o simulando una con ABC.

\begin{lstlisting}[caption={Interfaz para servicios de pago}]
from typing import Protocol

class Pagable(Protocol):
    def procesar_pago(self, monto: float) -> str: ...

class TarjetaCredito:
    def procesar_pago(self, monto: float) -> str:
        return f"Pago de {monto} con tarjeta de credito"

class PayPal:
    def procesar_pago(self, monto: float) -> str:
        return f"Pago de {monto} via PayPal"

# Ambas clases cumplen con el protocolo Pagable
metodos_pago: list[Pagable] = [TarjetaCredito(), PayPal()]

for metodo in metodos_pago:
    print(metodo.procesar_pago(150))
\end{lstlisting}

\section{Method Resolution Order (MRO)}

El MRO es la manera en la que Python busca metodos y atributos en una jerarquia de herencia, sobretodo en herencia multiple.

\begin{lstlisting}[caption={MRO resolviendo herencia múltiple}]
class A:
    def metodo(self): print("A")

class B(A):
    def metodo(self): print("B")

class C(A):
    def metodo(self): print("C")

class D(B, C): pass

d = D()
d.metodo()  # Salida: B (porque B esta primero)
\end{lstlisting}

\textbf{Orden de búsqueda:}
\[
D \rightarrow B \rightarrow C \rightarrow A \rightarrow object
\]

\section{Conclusión}

Los conceptos repasados en este informe crean en conjunto un paradigma de programación que permite crear codigo modular, reutilizable y escalable. La flexibilidad que permiten estos conceptos en su uso conforman una gran herramienta para aquellos que desean construir software robusto.

Las posibilidades que nos abre al estudiar y practicar constantemente estos conceptos nos permite, en conclusión, ser mejores desarrolladores.

\end{document}%

�ltima

modificaci�n:

para

PR

